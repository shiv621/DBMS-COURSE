\documentclass[10pt,a4paper,twoside]{article}
\usepackage[dutch]{babel}
\usepackage{amssymb}
\usepackage{amsmath}
\usepackage{float,flafter}	
\usepackage{hyperref}
\usepackage{inputenc}
\usepackage{graphicx}
\setlength\paperwidth{20.999cm}\setlength\paperheight{29.699cm}\setlength\voffset{-1in}\setlength\hoffset{-1in}\setlength\topmargin{1cm}\setlength\headheight{12pt}\setlength\headsep{0cm}\setlength\footskip{1.131cm}\setlength\textheight{25cm}\setlength\oddsidemargin{2.499cm}\setlength\textwidth{15.999cm}

\begin{document}
\begin{center}
\hrule

\vspace{.4cm}
{\bf {\Large TERM PAPER - UPDATES }}\\
\vspace{0.3cm}
{\bf {\huge DBMS FOR BREAST CANCER 


PATIENTS}}
\vspace{0.3cm}
\end{center}
{\bf Name:}  Shivam Yadav\\
{\bf Roll no:}  19111054 \\
{\bf Branch: }  Biomedical Engineering \hspace{\fill}  21 January, 2022 \\
\hrule

\vspace{.5cm}

\section{ABSTRACT}

Breast cancer patients' data was analysed utilising a new upgraded computerised database based on breast cancer risk variables such as age, race, breastfeeding, hormone replacement therapy, family history, and obesity.system of management MySQL (My Structural Query Language) is chosen as the database administration application.a method for storing patient data collected from Malaysian hospitals Incorporated into the programme is an automatic calculating tool.
This technology is designed to aid with data analysis. The findings are automatically plotted, and a user-friendly graphical user interface is provided.is being created to govern the MySQL database. Breast cancer is most common among women, according to case studies.Malay women are the most common, followed by Chinese and Indian women.Breast cancer is most common between the ages of 50 and 59.
The findings imply that the risk of breast cancer is higher in older women and lower in women who breastfeed. Weight status may have a different impact on breast cancer risk. More research is needed to corroborate these findings.

\section{ INTRODUCTION}

Many hospitals throughout the world have established a computerised database management (CDM) system to provide for proper management of medical records for various types of cancer patients (Ann et al. 2003). Due to the expensive installation and implementation costs, as well as a shortage of experienced technicians for maintenance, only a few hospitals in Malaysia have adopted a CDM system.Medical data is required in any health-care institution in order to avoid medical errors and erroneous decisions. To meet the growing demand for medical information accessibility, CDM-based management solutions have been created.
\vspace{0.3cm}

The goal of this study is to develop a CDM system that is appropriate for breast cancer patients in Malaysian hospitals, though the product can be modified to other patients or countries. The first section of the paper explains how to set up an analytical database management system. The system includes features including storing and retrieving patient data, inputting new patient data, updating or removing data, and scheduling appointments.For this, an automated computation tool is being developed in the analytic database management system. Age, race, breastfeeding, hormone replacement therapy, family history, and obesity are all investigated as breast cancer risk factors.

\vspace{0.3cm}
Microsoft Access and MySQL are the tools that can be
used to implement the relational database management
system. Relational database is a collection of data items
where the data are organized into the table form, and
data can be accessed in many different ways without
reorganizing the database tables (Allen 2006). This
database management system has the capability to gather,
store and transmit the medical record information from
different sites of hospital to a centralized database system
Microsoft Access is a well-known data management programme that allows you to store information or data in tables that it manages from your local hard drive (Paul 2011).
Microsoft Access can create a 'back-end' database that holds the needed data while still providing a user-friendly 'front-end' interface.Microsoft Access is a well-known data management tool that lets you store data or information in tables that it manages from your local hard disc (Paul 2011).
Microsoft Access can establish a 'back-end' database that stores the necessary information while also providing a user-friendly 'front-end' interface.

\section{ISSUES IN EXISTING PRACTICE IN MALAYSIA}

There are a lot of patients seeking diagnosis and medical
treatment of breast cancer in hospitals every day. As an
example, at the Melaka General Hospital (MGH), the
current practice is to handle the huge amount of data
through the hardcopy format.The patient registration process begins with a hardcopy format. If visible symptoms are identified, the physician will undertake a breast examination and recommend a relevant breast screening test such as a mammography, ultrasound, or breast biopsy. A hardcopy report containing a description of the patient's condition and the type of screening test is submitted to the hospital's radiology department.. Following the test, the photos and results, which are also available in hardcopy format, will be sent to the physician, who will then decide on the next course of action. The documentation area houses all of the hardcopies.
\vspace{0.3cm}
Apart from security considerations, physical transfer  reports is time consuming. When a medical report is required quickly, this might be a significant issue. In Furthermore, due to the high amount of data recorded,intricate layout in the documentation room of records that can obstruct retrieval, updating, or modification It's difficult and time-consuming to keep track of the records.Furthermore, when medical reports are lost or damaged, it is almost impossible to retrieve the lost information.
Unlike a computerised system which can perform routine partial checks, human errors incurred in filling the medical forms and reports are more difficult to be detected by the system, since the report must be seen by another person before the abnormally can be discovered.Furthermore, when medical reports are lost or damaged, it is practically hard to retrieve the lost information.
Unlike a computerised system which can run routine partial checks, human errors made in filling the medical forms and reports are more difficult to be caught by the system, since the report must be seen by another person before the abnormally can be discovered

\section{ AN ANAYLTIC DATABASE MANAGEMENT SYSTEM FOR BREAST
CANCER PATIENTS - THE OBJECTIVE}

An analytic database management system is presented to solve the constraints of the existing system, allowing data to be conveniently collected, saved, updated, and retrieved. The recommended solutions to address the mentioned challenges are detailed in Table 1.

\section{ Details of design}

The system is specifically built for easily and comfortably processing and organising medical information such as personal details and diagnosis reports used in breast cancer departments, as well as essential data analysis. Large databases are used to collect and preserve medical records.A graphical user interface (GUI) is a user-friendly 'front-end' interface that controls the 'back-end' database and acts as a communication link between the user and the database.
Table 2 lists the elements that have been incorporated into the database management system architecture.


\begin{figure}
  \includegraphics[width=1.1\linewidth]{1.png} \includegraphics[width=1.1\linewidth]{2.png}
  \includegraphics[width=1.1\linewidth]{3.png}
  
  
  
\end{figure}



\section{Data and database}

The type of data kept and the manner of data gathering have a big impact on database design. Patient information and diagnosis reports were used in this endeavour.are must be kept in the manner stated in Table 3.MGH provided a total of 1057 patient data for breast cancer (mostly in hardcopy format) for this project. A patient who undergoes an MRI examination has the data saved in the digital image and communications in medicine (DICOM) format, which is a standard for managing, storing, and transferring information in medical imaging. DICOM files can be shared between two entities, for example, to get images or obtain patient information.
\vspace{0.3cm}
Because the forms of the data differ, a standardized data format is required. A text file is created for data that is kept in hardcopy format in order to transform it.converting the patient's information to a softcopy format On the contrary,The contents of the DICOM file must be retrieved.and saved in a text file alongside the relevant MRI diagnosis report that has been converted to a softcopy version The DICOM files must be read in order to be read.As seen in the example, the MATLAB programming language is employed.1st Figure

\vspace{0.3cm}

The database is a method of storing and managing information that is methodical and well-organized.
Because of its low cost and appropriate security, MySQL was chosen as the essential system software to implement the system database. MySQL provides an authorised user with a fast, flexible, secure, and consistent means of retrieving, updating, and entering data into the database.
Using a structured query language (SQL) and statements that are part language, part mathematics, the developer can construct and manipulate the data in any way they choose.
The output of the LOAD DATA command is shown in Figure 3. The text file's data has been successfully transmitted.

\begin{figure}
  \includegraphics[width=1.1\linewidth]{22.png}
    \includegraphics[width=1.1\linewidth]{21.png}
  
  \includegraphics[width=1.1\linewidth]{20.png}
  
  
  
\end{figure}

\section{Concept of GUI}

The graphical user interface (GUI) serves as a communication link between the user and the database, allowing users to interact with the database with ease. For example, by simply clicking on specific features in the GUI such as a button or a checkbox, the user can easily update records, insert new records, or delete existing records in the MySQL database. There is no requirement to learn programming. Figure 4 depicts the notion of a graphical user interface.
\vspace{0.3cm}

On the left, there is a list of tools that you can use.
To design the GUI, drag the desired tools from the toolbox to the form side, such as a button, text box, group box, and list view. Figure 5 shows a list of the tools that can be found in the toolbox.
The Window Application Form 1 has a properties box in the lower right corner. It allows you to change the size, text, font, colour, and visibility of the tool. Figure 6 depicts the properties box that occurs after dragging a button onto the form side.

\begin{figure}
  \includegraphics[width=1.1\linewidth]{19.png}
    \includegraphics[width=1.1\linewidth]{18.png}
  
  \includegraphics[width=1.1\linewidth]{17.png}
  
  
  
\end{figure}
\section{Main menu page}
The main page has a collection of connections to several pages, such as the patient registration page, personal information, diagnosis report, statistical analysis, and appointments.
The level of access to the pages is determined by the user type. For example, a nurse or staff member can view the information, but only the physician can add or change the patient report.

\section{ Design of assessibility}

By selecting the appropriate button, the user can navigate to the desired page. The main menu page is made up of buttons that are grouped and framed within a group box. The visibility of the group box for the main menu feature is initially set to "false" in the properties box. The visibility is adjusted to "true" once access is granted.

\section{Patient registration}

Only the nurse or staff has access to this page, which is used to enter new patient records into the database. When vital data is not inserted, a message is sent to decrease human mistake and prevent data loss. There are various checkboxes that allow the user to choose the patient's background and breast cancer symptoms. If the button is activated, it sends all of the entered data to the MySQL database. To create the new patient registration page, all input controls are enclosed with a group box.

\section{Codes}

The visibility of the group box in the properties box is initially set to false. When the user clicks the button to access this page, the group box visibility is set to true in the codes, whereas the group box visibility for the main menu page is set to false. The patient's data is entered into the appropriate input text boxes and checkboxes. When the insert button is pressed, a series of actions are initiated. The textboxes for entering some vital patient data are first checked. If any of the text boxes are left blank, a message box appears as a reminder to fill in the blanks. While waiting for the delayed response, a delay system is applied.

\begin{figure}
  \includegraphics[width=1.1\linewidth]{16.png}
  
  
  
\end{figure}

\section{Personal details and diagnosis report}

A further enhancement for the security of analytic database
management system is accomplished through recording
date and time that the patient data are updated or edited in
MySQL database and then displayed on the GUI. Figure 8
shows the patient personal details and diagnosis code flow

\begin{figure}
  \includegraphics[width=1.1\linewidth]{15.png}
  
  
  
\end{figure}

\section{Design}
The user can pick the patient's name from the list view to view, update, and edit functions of the patient's personal details and diagnosis report. The data from the selected patient will be displayed in the text boxes and check boxes.To make changes to the patient's personal information and the diagnosis report,Text boxes' contents can be changed or updated, and Checkboxes can be checked or unchecked. Following that,modifications The update button is used to save the information.information into the MySQL database that has been updated

\begin{figure}
  \includegraphics[width=1.1\linewidth]{14.png}
  
  
  
\end{figure}

\section{Breast cancer patient data}

The function enables for patient distribution depending on their age and race, as well as groupings based on their diagnosis reports. There is also information about nursing, family history, and hormone replacement therapy. According to the body mass index determined using Eq. (1) and guided by Table 5, the patients' weight status is divided into three categories: normal weight, overweight, and obesity.

\section{Client network}
The analytic database management system is placed on various computers throughout the hospital to allow for multiple accesses and task execution. For example, at the patient registration counter, the analytic database management system is used to enter new patient information into the database. After completing the registration process, the doctor can access the new patient information from the system installed on his or her PC. A centralised MySQL database system allows multiple computers to have direct network access to the database for storing, retrieving, and updating data.
As shown in Figure 10, this is accomplished by setting up a computer as the MySQL database server.

\vspace{0.3cm}

The MySQL database server machine and each of the clients can be readily joined together using wireless local area network (LAN) technology. The unique IP address of the server is declared in the system programming portion loaded in each client, allowing access to the MySQL database on the server.

\begin{figure}
  \includegraphics[width=1.1\linewidth]{13.png}
  
  
  
\end{figure}

\section{Breast cancer data analysis}

The patient page, which includes the patient's personal information and diagnosis, is linked to a new patient registration page that clinicians can access. The MySQL database will not be updated if the nurse or staff edits the reports. The date and time of the update operation will be captured and stored in the MySQL database automatically. Overall analysis, breast cancer patient analysis, benign breast alterations patient analysis, and screening method analysis are the four analysis subjects available.

\section{Overall analysis}
The overall analysis is displayed when you click the overall analysis button, as seen in Figure 11.
The total analysis looks at a variety of characteristics, including age, race, breastfeeding, family history, and hormone replacement therapy, for all patients, regardless of whether or not they get breast cancer.
All of these variables are considered changeable and unchangeable factors that influence a patient's breast cancer risk.

\begin{figure}
  \includegraphics[width=1.1\linewidth]{12.png}
  
  
  
\end{figure}
\section{Breastfeeding}
Figure 12 demonstrates that 63 percent of the 1057 patients breastfed their infants, while 395 patients do not.
533 of the 663 patients are Malay. This result may reflect the fact that, in comparison to other races, the majority of Malays breastfed their infants.
\begin{figure}
  \includegraphics[width=1.1\linewidth]{11.png}
  
  
  
\end{figure}

\section{Family history}
A total of 111 patients having a family history of breast cancer were found among the 1057 patients. This is determined during the registration interview; patients who match the criteria outlined above are considered to have a family history of breast cancer.
There is no link between race and family history because family history is an unchangeable component that is strongly dependent on inherited DNA. There is currently no known link between the two


\begin{figure}
  \includegraphics[width=1.1\linewidth]{10.png}
  
  
  
\end{figure}

\section{Age}
Figure 14 shows that 83 percent of the 871 patients seeking medical advice on breast cancer are between the ages of 50 and 59. This could be linked to a decrease in female hormone production following menopause in this age range.
\begin{figure}
  \includegraphics[width=1.1\linewidth]{9.png}
  
  
  
\end{figure}
\section{Breast cancer analysis}
As previously indicated, depending on the results of a breast biopsy, there is a GUI feature that can discriminate between breast cancer and benign breast alteration. Based on the examination of breast biopsy results, 71 out of 1057 patients have been diagnosed with breast cancer. Breastfeeding, age, and race may all play a role in the discovery of relevant information that can help with breast cancer prevention and awareness. Discussions and outcomes for

Following that, each risk factor in patients with breast cancer is examined.

\section{Breastfeeding}

Figure 15 reveals that 32 of the breast cancer patients' infants were breastfed. The difference in the number of patients with and without breastfeeding is quite modest, indicating that breastfeeding and breast cancer risk are inversely related. This finding is consistent with the findings of various research looking into the influence of breastfeeding on the risk of breast cancer (Hisham and Cheng 2003; Yip and Ng 1996; Su et al. 2010). In order to support the analysis on the association between breastfeeding and breast cancer risk, more information such as breastfeeding length, milk adequacy, and the age of first breastfeeding can be obtained in the interview portion during the patient registration.

\begin{figure}
  \includegraphics[width=1.1\linewidth]{8.png}
  
  
  
\end{figure}

\section{Family history}

Among the total of 71 breast cancer patients, six have a family history of the disease. These people are classified as having a high risk of breast cancer.
\section{Breast cancer patients analysis – hormone replacement
therapy}
Only 5 individuals with breast cancer have agreed to hormone replacement therapy. Although several studies claimed that hormone replacement therapy increased the incidence of breast cancer, this claim could not be confirmed due to the limited number of patients involved.



\section{Breast cancer patients analysis – age risk factor}

Breast cancer is most common in women aged 50–59, according to Figure 16. This reflects the fact that Asians (60 to 65 years old) have a higher rate of breast cancer than Caucasians.
Older women are also more susceptible to breast cancer, according to the findings. Menopause is also thought to play a role.

\begin{figure}
  \includegraphics[width=1.1\linewidth]{7.png}
  
  
  
\end{figure}
\section{Breast cancer patients analysis – obesity risk factor}

Obesity may be one of the breast cancer risk factors.
Only 15 breast cancer patients' height and weight were provided by the hospital authority out of a total of 71. Twelve of the 15 patients are obese. The bulk of the 12 obese individuals are between the ages of 50 and 59. According to the findings, an increase in body mass increases the risk of breast cancer in older patients (>35 years old), which is in line with a previous study (Erin et al. 2010). The findings may aid in raising women's awareness of the need of maintaining a healthy weight.
\section{Analysis of benign breast changes}
Although benign breast alterations are not malignant, some forms, such as atypical lobular and ductal hyperplasia, do raise the risk of developing cancer. There are 937 patients with benign breast alterations among the 1057. The analysis page is depicted in Figure 17.

\section{Benign breast change analysis – breastfeeding}
Breastfeeding was common among patients with benign breast alterations, with 66 percent having done so. There is a significant difference between those who are breastfeeding and those who are not.
\section{Benign breast change analysis – family history}
There are a tiny proportion of patients with benign breast alterations who have a family history (11 percent ). This finding could indicate that benign breast alterations are unrelated to a family history of breast cancer.
\section{Benign breast change analysis – hormone replacement
therapy}


\begin{figure}
  \includegraphics[width=1.1\linewidth]{6.png}
  
  
  
\end{figure}
\section{Benign breast change analysis – race}
The majority of patients with benign breast alterations are Malay, according to the race distribution analysis in Figure 18. (59 percent ).
The percentage of Chinese women who have benign breast alterations is half that of Malay women. The lowest ethnic group is the Indian (11 percent ).
\begin{figure}
  \includegraphics[width=1.1\linewidth]{5.png}
  
  
  
\end{figure}


\section{Benign breast changes analysis – age}
The majority of people with benign breast alterations are between the ages of 50 and 59. Figure 8 indicates that a decrease in female hormone production before or after menopause tends to create changes in the physical appearance and internal activity of the breast.

\section{Screening methods analysis}
For the objective of breast cancer detection, four breast examination procedures have been developed: mammography screening, ultrasound screening, breast biopsy, and MRI screening.
Mammography screening is performed on 1003 of the 1057 patients. If cancer is suspected, another screening test is required because mammography is not optimal. This could be a mammogram, an MRI, or a biopsy of the breast. The results are shown in Figure 19.
\vspace{0.3cm}
There are 55 patients who have been classified as having a high risk of breast cancer. They will be checked with a breast MRI rather than a mammography screening test. If the results of the MRI screening test are still inconclusive, a breast biopsy or ultrasound is performed.
This study gives helpful information that may aid in the reduction of screening costs. Before deciding on a screening modality, factors such as breast anatomy (dense breasts lower mammography sensitivity) and patient age (age 35 to 50 can be screened with MRI) should be considered.
This is because, if the proper screening procedure is used, an accurate diagnosis can be acquired, saving the cost of additional tests.
\vspace{0.3cm}
There are 55 patients who have been classified as having a high risk of breast cancer. They will be checked with a breast MRI rather than a mammography screening test. If the results of the MRI screening test are still inconclusive, a breast biopsy or ultrasound is performed.
This study gives helpful information that may aid in the reduction of screening costs. Before deciding on a screening modality, factors such as breast anatomy (dense breasts lower mammography sensitivity) and patient age (age 35 to 50 can be screened with MRI) should be considered.
This is because, if the proper screening procedure is used, an accurate diagnosis can be acquired, saving the cost of additional tests.
\begin{figure}
  \includegraphics[width=1.1\linewidth]{4.png}
  
  
  
\end{figure}
\section{Conclusions}
This project proposes and develops an analytic database management system. This system not only provides the tools for storing, accessing, and updating breast cancer patient data, but it also gives the tools for performing data analysis on the stored data. 1057 breast cases data from General Hospital Melaka were collected and put in the new system in order to construct the analytic database management system. A graphical user interface (GUI) has been developed to give a user-friendly interface for controlling the database management system. The graphical user interface serves as a link between the user and the database. On the GUI interface, functions such as update, edit, remove, and data analysis had been defined.
\vspace{0.3cm}
The investigation yielded some interesting results. There are 71 people diagnosed with breast cancer among the 1057 patients, the bulk of whom are Malay, followed by Chinese and Indian. The majority of breast cancer patients are between the ages of 50 and 59. This finding indicates that as women get older, their chances of developing breast cancer increase. Breastfeeding patients make up a lesser percentage of these 71 breast cancer patients than non-breastfeeding individuals. This suggests that breastfeeding may aid in the prevention of breast cancer.
Other risk variables are also examined. The results of the study based on the types of screening tests performed on all patients have been compiled.
have been made possible by the creation of an analytic database management system.




\end{document}